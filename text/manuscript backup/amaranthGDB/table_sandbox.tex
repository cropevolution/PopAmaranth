\documentclass[9pt,twocolumn,twoside]{celabRxiv}
% Use the documentclass option 'lineno' to view line numbers
\setlength{\marginparwidth}{2cm}
\usepackage[textsize=tiny,colorinlistoftodos]{todonotes} % comments in margin
\usepackage[utf8]{inputenc}
\definecolor{cornflowerblue}{rgb}{0.39, 0.58, 0.93}
\usepackage{blindtext}
 \usepackage{longtable}

%%%%%%%Add comments in color
\newcommand{\mgs}[1]{{\small \textcolor{green}{#1}}} 
\newcommand{\jgd}[1]{{\small \textcolor{red}{#1}}}
\newcommand{\citex}[1]{{\small \textcolor{red}{CITE(#1)}}}
\newcommand{\X}{{\textcolor{red}{X}}}

\newcolumntype{b}{X}
\newcolumntype{s}{>{\hsize=.5\hsize}X}

% Set supplement numbers to S and start counting newly
\newcommand{\beginsupplement}{%
 \setcounter{table}{0}
 \renewcommand{\thetable}{S\arabic{table}}% 
 \setcounter{figure}{0}
 \renewcommand{\thefigure}{S\arabic{figure}}%
 }
 
  
 
\usepackage{hyperref}

\title{PopAmaranth: A population genetic genome browser for grain amaranths and their wild relatives}

\author[$\ast$,1]{José Gonçalves-Dias}
\author[$\ast$,$\ddagger$,1]{Markus G Stetter}
 

\affil[$\ast$]{Dept. of Plant Sciences, University of Cologne, Cologne, Germany}
\affil[$\ddagger$]{Cluster of Excellence on Plant Sciences, University of Cologne, Cologne, Germany}

 
\keywords{Amaranthus, Orphan Crop, Sustainable Agriculture, Genome Browser, Population genetics}

\runningtitle{PopAmaranth} % For use in the footer

%% For the footnote.
%% Give the last name of the first author if only one author;
% \runningauthor{FirstAuthorLastname}
%% last names of both authors if there are two authors;
 \runningauthor{Gonçalves-Dias and Stetter}
%% last name of the first author followed by et al, if more than two authors.
%\runningauthor{One \textit{et al.}}


\setboolean{displaycopyright}{true}

\begin{document}

\onecolumn
 This is a test for long text to confirm if one column is operative. The quick brown fox jumps over the lazy dog


\small
% \begin{longtable}[l]{*{3}{>{\arraybackslash}p{0.3\textwidth}}}
% \begin{longtable}[l]{*{3}\raggedright{{0.1}p{0.1}}p{0.5}}
%  \begin{longtable}[l]{*{3}{>{\arraybackslash}p{0.3\textwidth}}}
%  \begin{longtable}{lll{2cm}p{4cm}{3.5cm}}
\begin{longtable}{p{2cm}p{5cm}p{9cm}}



% \setlength\LTcapwidth{\columnwidth} % default: 4in (rather less than \textwidth...)
% \setlength\LTleft{0pt}            % default: \parindent
% \setlength\LTright{0pt}           % default: \fill

\caption{A simple longtable example}\\
\hline
\textbf{ Track Name} & \textbf{Description} & \textbf{Detailed explanation} \\\hline
\endfirsthead
\multicolumn{3}{l}%
{\tablename\ \thetable\ -- \textit{Continued from previous page}} \\
\hline
\textbf{ Track Name} & \textbf{Description} & \textbf{Detailed explanation} \\
\hline
\endhead
\hline \multicolumn{3}{r}{\textit{Continued on next page}} \\
\endfoot
\hline
\endlastfoot
\hline
\textbf{Annotation} & & \\
\hline
Reference Genome v2.0 & \textit{Amaranthus hypochondriacus} reference genome v2.0 \citep{lightfoot2017single} & Reference sequence of the A. hypochondriacus (version 2.0 from Phytozome) \citep{goodstein2012phytozome} \\
Gene Annotation v2.1 & \textit{Amaranthus hypochondriacus} gene annotation with subfeatures. & Follows Phytozome nomenclature for gene names. Clicking on the track, information for subfeatures CDS, mRNA, and UTR’s is present. For each of the subfeatures, its name, type, position on the chromosome, and length are described. \textit{Gene density} viewable from whole chromosome perspective. \\
\textbf{Differentiation} & & \\
\hline
F$_{st}$ & Weir-Cockerham pairwise F$_{st}$ \citet{weir1984estimating} & pairwise F$_{st}$ between species pairs calculated on non-overlapping 5 kb windows. Windows with F$_{st}$ below the mean are represented in red and above the mean in blue. Shading in light and darker grey represents 1$\sigma$ and 2$\sigma$ from the mean, respectively. A red window indicates a region with higher fixation of alleles (genetic differentiation) between the two species, and the opposite for a blue window. The yellow line represent the global mean F$_{st}$. %If two populations have a F$_{st}$ of 1, it means the populations are fixed for different alleles. 
For visualization purposes, scale is adjusted for the local values of the region in display.\\
& & 	\textit{Summary F$_{st}$}: Color gradient showing F$_{st}$ values for all comparisons in a compact way. Color scale varies from white (0) to darker blue (1). Each bar represents the value of one 5kb window. \\
\textbf{Diversity} & & \\
\hline
Wu \& Watterson $\theta$ & Estimator of Genetic diversity on population \citet{watterson1975number}& $\theta$ for each pairs calculated on non-overlapping 5 kb windows. Windows with $\theta$ below the mean are represented in red and above the mean in blue. Shading in light and darker grey represents 1$\sigma$ and 2$\sigma$ from the mean, respectively. The yellow line represent the global mean $\theta$. The lower the $\theta$, the lower the genetic diversity within the population. For visualization purposes, scale is adjusted for the local values of the region in display. \\
& & 	\textit{Summary Wu \& Watterson $\theta$}: Color gradient showing $\theta$ values for all comparisons in a compact way. Color scale varies from white (0) to darker blue (1) . Each bar represents the value of one 5kb window.\\
expected heterozygosity	& Expected rate of heterozygosity for each variant under Hardy-Weinberg equilibrium & SNP's expected heterozygosity are represented in blue when above the mean and in red when expected heterozygosity is below the mean. The yellow line represent the global mean expected heterozygosity. Values range between 0 and 1. For visualization purposes, scale is adjusted for the local values of the region in display. \\
& & 	\textit{Summary expected heterozygosity}: Color gradient showing expected heterozygosity values for all comparisons in a compact way. Color scale varies from white (0) to darker blue (1) . Each bar represents the value for each variant.\\
observed heterozygosity	& Rate of observed heterozygosity for each variant. & SNP's observed heterozygosity are represented in blue when above the mean and in red when observed heterozygosity is below the mean. The yellow line represent the global mean observed heterozygosity. Values range between 0 and 1. For visualization purposes, scale is adjusted for the local values of the region in display .
Each bar represents the value for each variant.\\
& & 	\textit{Summary observed heterozygosity}: Color gradient showing observed heterozygosity values for all comparisons in a compact way. Color scale varies from white (0) to darker blue (1) . Each bar represents the value for each variant.\\
inbreeding coefficient	& Calculate the inbreeding coefficient for each variant. & It is calculated based observed heterozygosity expected heterozygosity. SNP's inbreeding coefficient are represented in blue when above the mean and in red when the inbreeding coefficients below the mean. The yellow line represent the global mean inbreeding coefficient. Values range between 0 and 1. For visualization purposes, scale is adjusted for the local values of the region in display.\\
& & 	\textit{Summary inbreeding coefficient}: Color gradient showing inbreeding coefficient values for all comparisons in a compact way. Color scale varies from white (0) to darker blue (1) . Each bar represents the value for each variant.\\
nucleotide diversity ($\pi$) & Nucleotide diversity \citet{nei1979mathematical} & Windows of 5kb with average number of nucleotide differences above the mean are represented in blue and are represented in red when $\pi$ is below the mean. The yellow line represent the global mean $\pi$. Shading in light and darker grey represents 1$\sigma$ and 2$\sigma$ from the mean, respectively. For visualization purposes, scale is adjusted for the local values of the region in display. A track is available for each species.\\
& & 	\textit{Summary nucleotide diversity ($\pi$}: Color gradient showing $\pi$ values for all comparisons in a compact way. Color scale varies from white (0) to darker blue (1) . Each bar represents the value for each variant.\\

\textbf{Selection} & & \\ 
\hline
Tajima's D & difference between the mean number of pairwise differences and the number of segregating sites \cite{tajima1989statistical} & Deviations from neutral state (Tajima's D=0) are possible signals of selection or demographic changes.  
%Tajima’s D <0, describes an excess or rare alleles from a selective sweep or population expansion after a recent bottleneck occurred. Tajima’s D > 0 describes  a lack  of rare alleles and indicates balancing selection or a sudden population contraction. Tajima’s D = 0 means that the population is under mutation-drift equilibrium. 
Windows of 5kb with Tajima's D above the mean are represented in blue and in red when the values are below the mean . A track is available per species. Shading in light and darker grey represents 1$\sigma$ and 2$\sigma$ from the mean, respectively. The yellow line represent the global mean Tajima's D. For visualization purposes, scale is adjusted for the local values of the region in display.\\
& & 	\textit{Summary Tajima's D}: Color gradient showing Tajima's D values for all comparisons in a compact way. Color scale varies from red (<0) to white (0) to darker blue (1). Each bar represents the value of one 5kb window. \\
relative nucleotide diversity & Ratio of nucleotide diversity between a domesticated species and A. hybridus wild ancestor & Ratio between the nucleotide diversity of two amaranth species.  Tracks are available for each of the grain amaranth species (\textit{A. caudatus}, \textit{A. cruentus}, \textit{A. hypochondriacus}) comparatively to the wild ancestor (A. hybridus). Windows of 5kb above the mean are represented in blue and windows below the mean are represented in red.  For visualization purposes, scale is adjusted for the local values of the region in display. \\

Selective Sweep (RAiSD ($\mu$) & $\mu$ statistic for selective sweep detection & Each bar represents the window center of regions with $\mu$ above the top 1\% threshold. This represent regions with signals of positive selection. These regions are represented in blue.\\
& & 	\textit{Summary Selective Sweep (RAiSD ($\mu$)}: Color gradient showing $\mu$ values for all comparisons in a compact way. Blue colored regions represent regions with signal of positive selection.\\
\textbf{Variant Call} & &\\
\hline
VCF & Single nucleotide polymorphisms (SNP) identified.& Variant calling was performed relative to A. hypochondriacus 2.0 reference genome\citep{lightfoot2017single}. Tracks are available for each of the amaranth species. Variants are identified in pie charts. Gold colored slices represent the percentage of nonvariant samples, green represents the percentage of heterozygous phenotypes found for that site.\\ 


\end{longtable}
% \end{center}
\end{document}


\end{document}
